\documentclass[11pt,letterpaper]{article}

% Required packages
\usepackage[margin=1in]{geometry}
\usepackage{fancyhdr}
\usepackage{lastpage}
\usepackage{enumerate}
\usepackage{enumitem}

% Essential packages for figures and subfigures
\usepackage{graphicx}
\graphicspath{{.},{./figures},{./common},{./common/figures},{../common},{../common/figures}}
\usepackage[margin=.5ex]{subcaption}
\usepackage{float}
\usepackage[all]{hypcap}
\usepackage[noabbrev,capitalise,nameinlink]{cleveref}
  \crefname{figure}{figure}{figures}
  \crefname{table}{table}{tables}
  \crefname{problem}{problem}{problems}
  \Crefname{figure}{Figure}{Figures}
  \Crefname{table}{Table}{Tables}
  \Crefname{problem}{Problem}{Problems}

% Import book style files directly
\usepackage{common/styles-tex/bookmathmacros}
\usepackage{common/styles-tex/booktikz}

% Simple counter for problems
\newcounter{problem}
\setcounter{problem}{0}

% Exam header and footer setup
\pagestyle{fancy}
\fancyhf{}
\renewcommand{\headrulewidth}{0.4pt}
\renewcommand{\footrulewidth}{0.4pt}

% Exam-specific information
\newcommand{\examtitle}{Exam 1}
\newcommand{\examdate}{8 October 2025}
\newcommand{\examtime}{80 minutes}
\newcommand{\coursename}{ME/EE 345 Mechatronics}
\newcommand{\instructorname}{Rico A. R. Picone}
\newcommand{\examversion}{A}

% Header and footer content
\lhead{\coursename}
\chead{\examtitle}
\rhead{Version \examversion}
\lfoot{\instructorname}
\cfoot{Page \thepage\ of \pageref{LastPage}}
\rfoot{\examdate}

% Simple exam header command (no title page)
\newcommand{\makeexamheader}{%
  \begin{center}
    {\Large \textbf{\examtitle}} \\[0.3cm]
    {\large \coursename\ --- Version \examversion} \\[0.2cm]
    {\normalsize \examdate\ --- \examtime} \\[0.2cm]
    {\normalsize \instructorname}
  \end{center}
  \vspace{0.3cm}

  \noindent\textbf{Name:} \rule{3in}{0.5pt}
  \vspace{0.5cm}

  \noindent\textbf{Instructions:} In class exam. Open notes, closed book, closed computer. Calculator is allowed. Partial credit may be given. Use your own paper. Write your name in the space provided.
  \vspace{0.5cm}
}

% Command for problem spacing
\newcommand{\problemspace}[1][1.5cm]{
  \vspace{#1}
}

% Command for answer boxes
\newcommand{\answerbox}[2][4cm]{
  \par\vspace{0.5cm}
  \noindent\textbf{Answer:}
  \framebox[#1][l]{\rule{0pt}{#2}}
  \par\vspace{0.5cm}
}

% Command for solution space
\newcommand{\solutionspace}[1][5cm]{
  \par\vspace{0.2cm}
  \noindent\textbf{Solution:}
  \par\vspace{#1}
}

% Import bookcolors at document begin to avoid spurious output in preamble
\AtBeginDocument{
  \input{common/styles-tex/bookcolors.sty}
}

\begin{document}

% Create simple header
\makeexamheader

% Begin problems

\stepcounter{problem}
\noindent\textbf{Problem \theproblem~(30 points).} For the RL circuit diagram below, perform a complete circuit analysis to solve for $v_{o}(t)$ if $V_S(t) = A$, where $A \in \mathbb{R}$ is a given but unspecified constant. Let $i_L(t)|_{t=0} = 0$~A. Hint: Solve a differential equation for $i_L(t)$.

\begin{center}
\begin{circuitikz}[]
  \draw
    (0,0) to[voltage source, v=$V_S$] (0,2)
    to[R=$R$, i=$ $] (2,2)
    to[L=$L$, i=$ $] (2,0)
    -- (0,0);
  \draw
    (2,2) to[short, -o] (3,2)
    to[open, v^=$v_o$, -o] (3,0)
    to[short] (2,0);
\end{circuitikz}
\end{center}
\problemspace

\stepcounter{problem}
\noindent\textbf{Problem \theproblem~(20 points).} For the circuit diagram below, solve for the steady-state current $i_R(t)$ if $V_S(t) = A \angle \phi$. \emph{Do} write $V_S$ and the impedance of each element in phasor/polar form. \emph{Do not} substitute $V_S$ or the impedance of each element into your expression for $i_R(t)$. Recommendation: Use a divider rule.

\begin{center}
\begin{circuitikz}[]
\draw
  (0,0) to[voltage source, v=$V_s$] (0,2)
  to[L=$L$, i=$ $] ++(2,0)
  to[C=$C$, i=$ $] ++(0,-2)
  -- (0,0);
\draw
  (2,2) to[short] ++ (1.5, 0)
  to[R=$R$, i=$ $] ++(0,-2)
  to[short] ++(-1.5,0);
% \draw
%   (3.5,2) to[short, -o] ++(1,0)
%   to[open, v^=$v_o$, -o] ++(0,-2)
%   to[short] ++(-1,0);
\end{circuitikz}
\end{center}
\problemspace

\stepcounter{problem}
\noindent\textbf{Problem \theproblem~(25 points).} \label{ex:v10}

In each of the figures of \cref{fig:ex:v10}, solve for the voltage $v_{10}$ across the $10\,\Omega$ resistor.
Use the assumptions in the associated caption.
Clearly justify each response.

\begin{figure}[H]
  \centering
  \quad
  \subcaptionbox{Treat $D$ as ideal}{%
  \begin{circuitikz}[]
    \draw (0,0)
    coordinate (o)
    to[voltage source,v=$5$\,V] ++(0,2)
    to[R=$10\,\Omega$] ++(2,0)
    to[full led,l=$D$,i=$ $] ++(0,-2)
    -- (o);
  \end{circuitikz}
  }\quad
  \subcaptionbox{$V_S = 3 e^{j \pi}$, $N = 1/10$}{%
  \begin{circuitikz}[]
    \draw (0,0) node [transformer core,yscale=1](T){}
          (T.A1) node[below] {$1$}
          (T.B1) node[below] {$2$}
          (T.base) node {$N$};
    \draw
      (T.A2) to (-2,|-T.A2)
      to[voltage source, v=$V_S$] (-2,|-T.A1)
      -- (T.A1);
    \draw
      (T.B1)
      -- (2,|-T.B1)
      to[R=$10\,\Omega$, i=$ $] (2,|-T.B2)
      -- (T.B2);
    \draw[-triangle 45]
      (.42,-.8) -> +(0,-.2);
    \draw[-triangle 45]
      (-.42,-.8) -> +(0,-.2);
  \end{circuitikz}
  }
  \subcaptionbox{$V_T = 0.7$ V, $K = 0.5$ mA$/\text{V}^2$}{%
  \begin{circuitikz}[]
    \draw (0,0) node[nmos](T){};
    \coordinate (Vs) at ($(T.drain)+(0,1.5)$);
    \draw (Vs) to[R,l_=$10\,\Omega$,i=$ $] (T.drain);
    \draw (T.source) --++(-2,0)
      node[ground](g){}
      to[voltage source,v=$3$ V] ++(0,1.5)
      -- ++(.75,0)
      coordinate (bar)
      -- (bar |- T.gate)
      -- (T.gate);
    \draw (g) --++(-1.5,0)
      coordinate (foo)
      to[voltage source,v=$10$ V] (foo |- Vs)
      -- (Vs);
  \end{circuitikz}
  }
  \quad
  \subcaptionbox{}{%
  \begin{circuitikz}[]
    \draw (0,0) node[op amp,yscale=-1](opamp){}
    (opamp.out) node[above] { }
    (opamp.out) -- ++(.75,0)
    to[R,label=$10\,\Omega$,i=$ $] ++(0,-1.5)
    node[ground]{}
    (opamp.out) -- ++(0,-1.5)
    coordinate (foo)
    to[R=$40\,\Omega$,i=$ $] (foo -| opamp.-)
    -- (opamp.-);
    \draw (-2,-1.5)
    node[ground](g){}
    to[voltage source,label=$5$ V] (g |- opamp.+)
    -- (opamp.+);
  \end{circuitikz}%
  }
  \caption{Circuits.\label{fig:ex:v10}}
\end{figure}
\problemspace

\stepcounter{problem}
\noindent\textbf{Problem \theproblem~(25 points).} Write a one- or two-sentence response to each of the following questions and imperatives.
The use of equations is acceptable when they appear in a sentence.
Simple diagrams are acceptable.
Don't quote me (use your own words, other than technical terminology).
\begin{enumerate}[label=(\alph*)]
  \item If the current through an inductor is suddenly switched off, what happens?
  \item Let the output voltage of a resistor circuit be $5$ V and the equivalent resistance $500\,\Omega$. What is the Thevenin equivalent circuit?
  \item In the preceding part of this question, what is the Norton equivalent?
  \item When can we use impedance analysis?
\end{enumerate}
\problemspace

\end{document}
